%%%%%%%%%%%%%%%%%%%%%%%%%%%%%%%%%%%%%%%%%%%%%%%%%%%%%%%%%%%%%%%%%%%%%%
% MATH 2007 - Calculus II Review
% by Simon Pratt
%%%%%%%%%%%%%%%%%%%%%%%%%%%%%%%%%%%%%%%%%%%%%%%%%%%%%%%%%%%%%%%%%%%%%%
\documentclass{article}

\usepackage{fullpage}
\usepackage{parskip}
\usepackage{amsmath}

\title{MATH 2007 Review}
\author{Simon Pratt}
\date{\today}

\begin{document}

{\Huge MATH 2007 Review}

\section{Topics Covered}
{\tiny *Section numbers are from Nelson's custom text book for the course}

Test 1
\begin{itemize}
\item 4.5 Integration by Substitution
\item 8.2 Integration by Parts
\item 8.3 Trigonometric Integrals
\item 8.4 Trigonometric Substitution
\end{itemize}

Test 2
\begin{itemize}
\item 8.5 Partial Fractions
\item 8.7 Indeterminate Forms and L'H\^{o}spital's Rule
\item 8.8 Improper Integrals
\item 10.2 Plane Curves and Parametric Equations
\end{itemize}

Test 3
\begin{itemize}
\item 10.3 Parametric Equations and Calculus
\item 10.4 Polar Coordinates and Polar Graphs
\item 10.5 Area and Arc Length in Polar Coordinates
\item 9.1 Sequences
\end{itemize}

Test 4
\begin{itemize}
\item 9.2 Series and Convergence
\item 9.3 The Integral Test and p-Series
\item 9.4 Comparisons of Series
\item 9.5 Alternating Series
\end{itemize}

Final Exam
\begin{itemize}
\item 9.6 The Ratio and Root Tests
\item 9.7 Taylor Polynomials and Approximations
\item 9.8 Power Series
\item 9.9 Representation of Functions by Power Series
\item 9.10 Taylor and Maclaurin Series
\end{itemize}

\newpage
\section{Integration by Substitution}

When we are presented with an integral that is the product of a
function with its derivative, we can substitute the function for a new
variable and its derivative for the differential of that variable.

For example:

\[
\int \frac{dx}{xlnx}
\]

We know the derivative of $lnx$ is $\frac{1}{x}$.  Let $u=lnx$,
$du=\frac{1}{x}$, then we have:

\[
\int u^{-1}du
\]

Then we can differentiate with respect to $u$ and get:

\[
ln \left| u \right| \\
\]

And substituting $u$ back in, we get:

\[
\int \frac{dx}{xlnx} = \ln \left| lnx \right|
\]

This method of integration is analagous to the chain rule for
differentiation.

\newpage
\section{Integration by Parts}

Sometimes we are presented with the integral of a product of
functions, one of which gets simpler when derived and the other gets
simpler when integrated.  In this case, we can use the formula:

\[
\int udv = uv - \int vdu
\]

For example:

\[
\int xe^xdx
\]

Let $u = x$, $dv = e^x$, and thusly $du = dx$, $v = e^x$, we get:

\begin{align*}
  &xe^x - \int e^xdx \\
  = &xe^x - e^x \\
  = &e^x(x-1)
\end{align*}

This method of integration is analagous to the product rule for
differentiation.

\end{document}